\chapter{\SECH--Browser}

\section{Projektziel}
Das Web besteht aus einer Vielzahl von Seiten mit unterschiedlichsten
Informationen. Betrachtet man diese Seiten genauer so stellt man fest,
dass ihr informationstragender Inhalt praktisch immer statisch ist.
Er wird Zeitpunkt der Erstellung der Web--Seite von einem Autor
festgelegt und bleibt bis zum n�chsten Update in genau diesem
Zustand. Auch wenn die Seiten dynamisch generiert werden, der daf�r
verwendete Inhalt ist vorher von einem Autor erstellt worden. Dass
diese Informationen statisch sind bedeutet auch, dass die Seiten f�r
alle Betrachter weltweit identisch sind.

Ziel des \SECH--Browsers ist es, diese statische Pr�sentation der
Informationen aufzuheben und die von einem Autor vorgegebenen Inhalte
automatisch um benutzerspezifische Informationen, den
\emph{\SECH--Annotations}\footnote{\SECH steht dabei f�r \textbf{S}elf
  \textbf{E}mbeding \textbf{C}haracteristic \textbf{H}yperlinks.}, zu
erg�nzen. Er folgt damit der Idee des \glqq taking the content to the
user\grqq\footnote{http://eexcess.eu} des EEXCESS--Projektes.

Die Auswahl der zus�tzlichen Informationen wird dabei von dem
\SECH--Browser, also dem Client, und nicht dem Server
vorgenommen. Personenbezogene Daten verlassen also nur dann den
eigenen Rechner, wenn es f�r die Suche nach den Daten der
\SECH--Annotations notwendig ist. Somit ist ein Maximum an
Datenschutz gegeben.

In der ersten Version basiert die Erzeugung der \SECH--Annotations
durch den \SECH--Brwoser auf Informationen, die der Seitenautor in
einem speziellen Format im Text hinterlegt hat. Diese Informationen,
die \SECH--Tags, verwendet der \SECH--Browser dann dazu, benutzerspezifische
Anfragen an eine Wissensquelle, beispielsweise dem Privacy--Proxy des
EEXCESS--Projektes (\url{http://eexcess.eu}), zu stellen und mit den
Ergebnissen die urspr�ngliche Darstellung der WWW--Seite zu
erg�nzen. 

In den weiteren Entwicklungsschritten soll der \SECH--Browser dann die
f�r die Erzeugung der \SECH--Annotations notwendigen Informationen
selbst�ndig aus den Textinhalten ableiten. Daf�r k�nnen dann
beispielsweise Techniken aus den Bereichen des Text-- beziehungsweise
Opinion--Minings verwendet werden.

\subsection{Anwendungsszenarien}
Im folgenden soll an Hand von einigen Beispielen dargestellt werden,
wie die zus�tzlichen Funktionalit�ten des \SECH--Browsers den Anwender
unterst�tzen k�nnen.

\subsubsection{Fremdenverkehr}
Web--Seiten von Fremdenverkehrsverb�nden werden von einer 
Vielzahl von verschiedenen Benutzergruppen besucht:
\begin{itemize}
     \item Besucher, die im n�heren Umkreis leben.
     \item Besucher von weiter her.
     \item Besuchern mit unterschiedlichen Muttersprachen.
     \item Besucher verschiedener Altergruppen.
     \item ...
\end{itemize}
Die verschiedenen Benutzergruppen haben in der Regel auch verschiedene
Anforderungen an die zur Verf�gung gestellten Informationen:
\begin{itemize}
     \item Ein Besucher, der nicht im n�heren Umkreis lebt, ben�tigt
    in der Regel eher Informationen �ber die Hotels des Gebietes als ein
    Besucher aus dem Umkreis.
     \item W�hrend Jugendliche sich eher f�r die Angebote aus dem Bereich
    der Fun--Sport--Arten interessieren, k�nnten Senioren eher an
    Informationen �ber die kulturellen Angebote der Region
    interessiert sein.
     \item Fremdsprachige Benutzer sollten vorrangig Informationen in
    ihrer Landessprache pr�sentiert werden.
\end{itemize}

Liegt ein gutes Design der Web--Seite vor, k�nnen die Besucher weitere Teile
der f�r sie relevanten Daten durch entsprechende Verlinkungen auf
andere Web--Seiten finden. Aber auch diese zus�tzlichen Informationen
sind statisch, f�r alle Besucher der Seite identisch und somit nicht
an die Bed�rfnisse der einzelnen Besucher angepasst.

Auch in diesem Anwendungsfall w�re es viel sinnvoller, wenn der
\SECH--Browser an Hand der ihm bekannten benutzerspezifischen
Informationen selbst�ndig erkennt, welche zus�tzlichen Informationen
f�r den Besucher interessant sein k�nnten und diese dann entsprechend
aufbereitet darstellen.

\subsubsection{Noch einer}

\section{Aufgabenstellung Wintersemester 2015/16}
Die im Wintersemester 2015/16 zu entwickelnde Version des
\SECH--Browsers umfasst folgende Schritte:
\begin{enumerate}
     \item Erstellung eines iOS--Programms zur Darstellung von
    Web--Seiten.
     \item Erstellung eines iOS/OS X Programms zur Generierung von
    Anfragen an den Privacy--Proxy des EEXCESS--Projekts und der
    Darstellung der jeweiligen Antworten.
     \item Definition der Syntax und Semantik des \SECH--Tags
    zur Beschreibung der vom Autor vorgegebenen Informationen zur
    Erstellung einer \SECH--Annotation.
     \item Erzeugung der um benutzerorientierte Informationen
    angereicherten Anfragen f�r den Privacy--Proxy.
     \item Integration der einzelnen Teilsysteme in einen lauff�higen
    Prototypen. 
\end{enumerate}
F�r die Software--Bausteine sind entsprechende Spezifikations--,
Konstruktions-- und Testdokumente zu erstellen. Diese Dokumente
orientieren sich an den Inhalten der Vorlesung \glqq Software
Engineering II\grqq.
 
Erweiterungen, wie beispielsweise
\begin{enumerate}
     \item Rating des Benutzers �ber die G�te der \SECH--Annotations einholen;
     \item Verwendung des Ratings um darauf folgende, �hnliche
    \SECH--Annotations zu verbessern;
\end{enumerate}
k�nnen in das Projekt einflie�en.