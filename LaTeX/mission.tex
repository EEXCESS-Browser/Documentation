\chapter{\SECH--Browser}

\section{Projektziel}
Das Web besteht aus einer Vielzahl von Seiten mit unterschiedlichsten
Informationen. Betrachtet man diese Seiten genauer so stellt man fest,
dass ihr informationstragender Inhalt praktisch immer statisch ist.
Er wird zum Zeitpunkt der Erstellung der Web--Seite von einem Autor
festgelegt und bleibt bis zum nächsten Update in genau diesem
Zustand. Auch wenn die Seiten dynamisch generiert werden, der dafür
verwendete Inhalt ist vorher von einem Autor erstellt worden. Dass
diese Informationen statisch sind bedeutet auch, dass die Seiten für
alle Betrachter weltweit identisch sind.

Ziel des \SECH--Browsers ist es, diese statische Präsentation der
Informationen aufzuheben und die von einem Autor vorgegebenen Inhalte
automatisch um benutzerspezifische Informationen, den
\emph{\SECH--Annotations}\footnote{\SECH steht dabei für \textbf{S}elf
  \textbf{E}mbeding \textbf{C}haracteristic \textbf{H}yperlinks.}, zu
ergänzen. Er folgt damit der Idee des \glqq taking the content to the
user\grqq\footnote{http://eexcess.eu} des EEXCESS--Projektes.

Die Auswahl der zusätzlichen Informationen wird dabei von dem
\SECH--Browser, also dem Client, und nicht dem Server
vorgenommen. Personenbezogene Daten verlassen also nur dann den
eigenen Rechner, wenn es für die Suche nach den Daten der
\SECH--Annotations notwendig ist. Somit ist ein Maximum an
Datenschutz gegeben.

In der ersten Version basiert die Erzeugung der \SECH--Annotations
durch den \SECH--Browser auf Informationen, die der Seitenautor in
einem speziellen Format im Text hinterlegt hat. Diese Informationen,
die \SEACH--Tags\footnote{\SEACH steht dabei für \textbf{S}elf
  \textbf{E}mbeding \textbf{A}nnotation based \textbf{C}haracteristic \textbf{H}yperlinks.}, verwendet der \SECH--Browser dann dazu, benutzerspezifische
Anfragen an eine Wissensquelle, beispielsweise dem Privacy--Proxy des
EEXCESS--Projektes (\url{http://eexcess.eu}), zu stellen und mit den
Ergebnissen die ursprüngliche Darstellung der WWW--Seite zu
ergänzen. 

In den weiteren Entwicklungsschritten soll der \SECH--Browser dann die
für die Erzeugung der \SECH--Annotations notwendigen Informationen
selbständig aus den Textinhalten ableiten. Dafür können dann
beispielsweise Techniken aus den Bereichen des Text-- beziehungsweise
Opinion--Minings verwendet werden.

\subsection{Anwendungsszenarien}
Im folgenden soll an Hand von einigen Beispielen dargestellt werden,
wie die zusätzlichen Funktionalitäten des \SECH--Browsers den Anwender
unterstützen können.

\subsubsection{Fremdenverkehr}
Web--Seiten von Fremdenverkehrsverbänden werden von einer 
Vielzahl von verschiedenen Benutzergruppen besucht:
\begin{itemize}
     \item Besucher, die im näheren Umkreis leben.
     \item Besucher von weiter her.
     \item Besuchern mit unterschiedlichen Muttersprachen.
     \item Besucher verschiedener Altergruppen.
     \item ...
\end{itemize}
Die verschiedenen Benutzergruppen haben in der Regel auch verschiedene
Anforderungen an die zur Verfügung gestellten Informationen:
\begin{itemize}
     \item Ein Besucher, der nicht im näheren Umkreis lebt, benötigt
    in der Regel eher Informationen über die Hotels des Gebietes als ein
    Besucher aus dem Umkreis.
     \item Während Jugendliche sich eher für die Angebote aus dem Bereich
    der Fun--Sport--Arten interessieren, könnten Senioren eher an
    Informationen über die kulturellen Angebote der Region
    interessiert sein.
     \item Fremdsprachige Benutzer sollten vorrangig Informationen in
    ihrer Landessprache präsentiert werden.
\end{itemize}

Liegt ein gutes Design der Web--Seite vor, können die Besucher weitere Teile
der für sie relevanten Daten durch entsprechende Verlinkungen auf
andere Web--Seiten finden. Aber auch diese zusätzlichen Informationen
sind statisch, für alle Besucher der Seite identisch und somit nicht
an die Bedürfnisse der einzelnen Besucher angepasst.

Auch in diesem Anwendungsfall wäre es viel sinnvoller, wenn der
\SECH--Browser an Hand der ihm bekannten benutzerspezifischen
Informationen selbständig erkennt, welche zusätzlichen Informationen
für den Besucher interessant sein könnten und diese dann entsprechend
aufbereitet darstellen.

\subsubsection{Noch einer}

\section{Aufgabenstellung Wintersemester 2015/16}
Die im Wintersemester 2015/16 zu entwickelnde Version des
\SECH--Browsers umfasst folgende Schritte:
\begin{enumerate}
     \item Erstellung eines iOS--Programms zur Darstellung von
    Web--Seiten.
     \item Erstellung eines iOS/OS X Programms zur Generierung von
    Anfragen an den Privacy--Proxy des EEXCESS--Projekts und der
    Darstellung der jeweiligen Antworten.
     \item Definition der Syntax und Semantik des \SEACH--Tags
    zur Beschreibung der vom Autor vorgegebenen Informationen zur
    Erstellung einer \SECH--Annotation.
     \item Erzeugung der um benutzerorientierte Informationen
    angereicherten Anfragen für den Privacy--Proxy.
     \item Integration der einzelnen Teilsysteme in einen lauffähigen
    Prototypen. 
\end{enumerate}
Für die Software--Bausteine sind entsprechende Spezifikations--,
Konstruktions-- und Testdokumente zu erstellen. Diese Dokumente
orientieren sich an den Inhalten der Vorlesung \glqq Software
Engineering II\grqq.
 
Erweiterungen, wie beispielsweise
\begin{enumerate}
     \item Rating des Benutzers über die Güte der \SECH--Annotations einholen;
     \item Verwendung des Ratings um darauf folgende, ähnliche
    \SECH--Annotations zu verbessern;
\end{enumerate}
können in das Projekt einfließen.