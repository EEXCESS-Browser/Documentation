\documentclass[a4paper,12pt]{article}
\usepackage[utf8]{inputenc}
\usepackage[german, ngerman]{babel}
\usepackage{graphicx}
\usepackage[figurename=Bild]{caption}
\usepackage{inputenc}
\usepackage{geometry}


%opening
\title{Constraints erstellen}
\author{Burak Erol, Philipp Winterholler}

\begin{document}

\section{Constraints erstellen}

Damit die Bedienelemente und das UI vom Browser auf verschiedenen Endger�ten identisch dargestellt wird, m�ssen Constraints gesetzt werden. Somit wird eine dynamische, und bei bestimmten Bedienelementen, eine feste Position gesetzt bzw. angepasst. Komplikationen mit den schon vorhandenen Constraints vom Container im Hintergrund waren vorhanden, somit mussten alle Constraints gel�scht und neu erstellt werden. Nachdem das WK-Webview dennoch keine H�he und Breite vom Container im Hintergrund angenommen hatte (Vermutung: Im main.storyboard nicht verwendete Constraints, die das Layout ungewollt �ndern), musste der Navigation Controller gel�scht und erneut integriert werden. Hierzu wurden folgende Schritte beachtet:

- Referenzen der Bedienelemente im main.storyboard und im ViewController l�schen 
- Men�leiste samt Buttons im main.storyboard l�schen 
- Tabelle/Zellen l�schen
- Container im Hintergrund l�schen
- Navigation Controller ausklinken und entfernen

Navigation Controller neu erstellen:
- Men�leiste samt Buttons einf�gen und feste Constraints setzen
- Container einf�gen und feste Constraints setzen
- Tabelle/Zellen einf�gen und feste Constraints setzen
- Referenzen neu verlinken und alles wieder in den ViewController integrieren

Nach erfolgreichem absolvieren dieser Schritte wurden im ViewController zwei Contraints erzeugt. Diese dienen f�r die Zuweisung der H�he und Breite vom Container auf den dar�ber liegenden WKWebView, welches programmatisch erzeugt wird. �ndern sich die Constraints vom Container, beispielsweise beim rotieren des Endger�tes, so �ndert sich auch H�he und Breite des WKWebViews. 

\includegraphics[width=12cm]{Pics/WKWebView_Hochformat}
\newpage
\includegraphics[width=12cm]{Pics/WKWebView_Querformat}

\end{document}
