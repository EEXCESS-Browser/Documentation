\chapter{Testen der Applikation}

\section{Einleitung}

Zum Testen der Applikation \SECH-Browser ist es erforderlich, die in einer exemplarisch erzeugten Webseite (www.sech-browser.de/tests) eingebetteten \SEARCH-Tags zu überprüfen. Um idealerweise alle Problemfelder bzw. Fehler der App erkennen zu können, ist es erforderlich eine bestmögliche Abdeckung aller möglichen Testfälle zu gewährleisten.  

\section{Testfälle} 

Um eine bestmögliche Überprüfung der App zu erreichen, wurde eine systematische Implementierung aller denkbaren Testfälle unter \url{http://www.sech-browser.de/old-index.html} vorgenommen. Wie aus der Definition der \SEARCH-Tags (s. WIKI des github-Projektordners) ersichtlich, besteht das \SEARCH-Tag aus den drei Teilen:

\begin {itemize} 
   \item \SEARCH-Head
   \item \SEARCH-Section
   \item \SEARCH-Link
\end {itemize}

Weiterhin kann jeder Teil des \SEARCH-Tags bis zu 5 Attribute enthalten:

\begin {itemize}
   \item topic
   \item type
   \item media-type
   \item provider
   \item licence
\end {itemize}

Um alle möglichen realisierten Kombinationen der Teile der \SEARCH-Tags mit den zugehörigen Attributen aufrufen zu können, wurde eine übersichtliche Menüstruktur gewählt. Zusätzlich wurden im letzten Menüpunkt 'Fehler' noch syntaktische Fehler in den Tags eingebaut, wie Weglassen von Tag-Klammern sowie Weglassen von Anführungszeichen bei Attributen oder falsch geschriebene Attributnamen.   
Die Menü-Struktur der Testfälle ist nachfolgend dargestellt:

\pagebreak

\centering \textbf{{\large Menü -- Testfälle}}

% ----------- Hauptmenu ------------------

% ---------- keine search-tags --------------------

\flushleft \textbf{keine \SEARCH-Tags}

% ---------- search-head --------------------

\flushleft \textbf{nur \SEARCH-Head}
\begin{enumerate}
   \item mit allen Attributen
   \item mit einzelnen Attributen  
   
   \begin{enumerate}  % einzelne Attribute        
        \item nur Topic  
        \begin{itemize}
           \item mit Umlauten
           \item mit Leerzeichen
           \item mit Sonderzeichen
           \item mit Ziffern
        \end{itemize}
     
        \item nur Type
        \begin{itemize}
           \item mit Umlauten
           \item mit Leerzeichen
           \item mit Sonderzeichen
           \item mit Ziffern
        \end{itemize}
	
        \item nur mediaType
        \begin{itemize}
           \item mit Umlauten
           \item mit Leerzeichen
           \item mit Sonderzeichen
           \item mit Ziffern
        \end{itemize}
  
        \item nur Provider
        \begin{itemize}
           \item mit Umlauten
           \item mit Leerzeichen
           \item mit Sonderzeichen
           \item mit Ziffern
        \end{itemize}
  
        \item nur Licence
        \begin{itemize}
           \item mit Umlauten
           \item mit Leerzeichen
           \item mit Sonderzeichen
           \item mit Ziffern
        \end{itemize}
    \end{enumerate} % einzelne Attribute
    \item mit mehreren Attributen
    \begin{enumerate}
        \item Topic, Type
	     \item Topic, Type, mediaType
	     \item Topic, Type, mediaType, Provider
     \end{enumerate}
  \end{enumerate} % search-head



\pagebreak

% ---------- search-section --------------------

\flushleft \textbf{nur \SEARCH-Section}

\begin{enumerate}
  \item mit allen Attributen
  \item mit einzelnen Attributen

   \begin{enumerate}
    \item nur Topic  
    \begin{itemize}
    \item mit Umlauten
    \item mit Leerzeichen
    \item mit Sonderzeichen
    \item mit Ziffern
\end{itemize}
	\item nur Type
 	\begin{itemize}
    \item mit Umlauten
    \item mit Leerzeichen
    \item mit Sonderzeichen
    \item mit Ziffern
\end{itemize}
	\item nur mediaType
	\begin{itemize}
    \item mit Umlauten
    \item mit Leerzeichen
    \item mit Sonderzeichen
    \item mit Ziffern
    \end{itemize}
    \item nur Provider
	\begin{itemize}
    \item mit Umlauten
    \item mit Leerzeichen
    \item mit Sonderzeichen
    \item mit Ziffern
    \end{itemize}
    \item nur Licence
	\begin{itemize}
    \item mit Umlauten
    \item mit Leerzeichen
    \item mit Sonderzeichen
    \item mit Ziffern
    \end{itemize}
\end{enumerate}
\item mit mehereren Attributen
\begin{enumerate}
    \item Topic, Type
	\item Topic, Type, mediaType
	\item Topic, Type, mediaType, Provider
    \end{enumerate}
\end{enumerate}

\pagebreak

% ---------- search-link --------------------

\flushleft \textbf{nur \SEARCH-Link}

\begin{enumerate}
  \item mit allen Attributen
  \item mit einzelnen Attributen

   \begin{enumerate}
    \item nur Topic  
    \begin{itemize}
    \item mit Umlauten
    \item mit Leerzeichen
    \item mit Sonderzeichen
    \item mit Ziffern
\end{itemize}
	\item nur Type
 	\begin{itemize}
    \item mit Umlauten
    \item mit Leerzeichen
    \item mit Sonderzeichen
    \item mit Ziffern
\end{itemize}
	\item nur mediaType
	\begin{itemize}
    \item mit Umlauten
    \item mit Leerzeichen
    \item mit Sonderzeichen
    \item mit Ziffern
    \end{itemize}
    \item nur Provider
	\begin{itemize}
    \item mit Umlauten
    \item mit Leerzeichen
    \item mit Sonderzeichen
    \item mit Ziffern
    \end{itemize}
    \item nur Licence
	\begin{itemize}
    \item mit Umlauten
    \item mit Leerzeichen
    \item mit Sonderzeichen
    \item mit Ziffern
    \end{itemize}
\end{enumerate}
\item mit mehreren Attributen
\begin{enumerate}
    \item Topic, Type
	 \item Topic, Type, mediaType
	 \item Topic, Type, mediaType, Provider
\end{enumerate}
\end{enumerate}

\pagebreak

% ------------- search-head und search-section ----------------------

\textbf{\SEARCH-Head und \SEARCH-Section} 
\begin{enumerate}
 	\item Standard
  	\item mit Umlauten
	\item mit Leerzeichen
	\item mit Sonderzeichen
	\item mit Ziffern
\end{enumerate}

% ------------- search-head und search-link ----------------------

\textbf{\SEARCH-Head und \SEARCH-Link} 
\begin{enumerate}
   \item Standard
   \item mit Umlauten
   \item mit Leerzeichen
   \item mit Sonderzeichen
   \item mit Ziffern
\end{enumerate}

% ------------- search-section und search-link ----------------------

\textbf{\SEARCH-Section und \SEARCH-Link} 
\begin{enumerate}
   \item Standard
   \item mit Umlauten
   \item mit Leerzeichen
   \item mit Sonderzeichen
   \item mit Ziffern
\end{enumerate}


% ------------- search-head, search-section und search-link ----------------------

\textbf{\SEARCH-Head, \SEARCH-Section, \SEARCH-Link}
\begin{enumerate}
   \item Standard
   \item mit Umlauten
   \item mit Leerzeichen
   \item mit Sonderzeichen
   \item mit Ziffern
\end{enumerate}

\pagebreak

\textbf{{\large Fehler}}

\textbf{nur \SEARCH-Head}
\begin{enumerate}
\item Anführungszeichen vorne fehlt
\item Anführungszeichen hinten fehlt
\item Tag-Klammer vorne fehlt
\item Tag-Klammer hinten fehlt
\end{enumerate}

\textbf{nur \SEARCH-Section}
\begin{enumerate}
\item Anführungszeichen vorne fehlt
\item Anführungszeichen hinten fehlt
\item Beginn Tag-Klammer vorne fehlt
\item Beginn Tag-Klammer hinten fehlt
\item Ende Tag-Klammer vorne fehlt
\item Ende Tag-Klammer hinten fehlt
\end{enumerate}

\textbf{nur \SEARCH-Link}
\begin{enumerate}
\item Anführungszeichen vorne fehlt
\item Anführungszeichen hinten fehlt
\item Beginn Tag-Klammer vorne fehlt
\item Beginn Tag-Klammer hinten fehlt
\item Ende Tag-Klammer vorne fehlt
\item Ende Tag-Klammer hinten fehlt
\end{enumerate}

\textbf{\SEARCH-Head und \SEARCH-Section} 
\begin{enumerate}
\item \SEARCH-Head Topic Anführungszeichen vorne fehlt
\item \SEARCH-Section Topic Anführungszeichen hinten fehlt
\item \SEARCH-Head Tag-Klammer vorne fehlt
\item \SEARCH-Section Beginn Tag-Klammer hinten fehlt
\item \SEARCH-Section Ende Tag-Klammer hinten fehlt
\item \SEARCH-Section nicht geschlossen
\end{enumerate}

\pagebreak
\textbf{\SEARCH-Head und \SEARCH-Link} 
\begin{enumerate}
\item \SEARCH-Head Topic Anführungszeichen vorne fehlt
\item \SEARCH-Link Topic Anführungszeichen hinten fehlt
\item \SEARCH-Head Tag-Klammer vorne fehlt
\item \SEARCH-Link Beginn Tag-Klammer hinten fehlt
\item \SEARCH-Link Ende Tag-Klammer hinten fehlt
\item \SEARCH-Link nicht geschlossen
\end{enumerate}

\textbf{\SEARCH-Section und \SEARCH-Link} 
\begin{enumerate}
\item \SEARCH-Section Topic Anführungszeichen vorne fehlt
\item \SEARCH-Link Topic Anführungszeichen hinten fehlt
\item \SEARCH-Section Tag-Klammer vorne fehlt
\item \SEARCH-Section Ende  Tag-Klammer hinten fehlt
\item \SEARCH-Section nicht geschlossen
\item \SEARCH-Link Beginn Tag-Klammer hinten fehlt
\item \SEARCH-Link Ende  Tag-Klammer hinten fehlt
\item \SEARCH-Link nicht geschlossen
\end{enumerate}

\textbf{\SEARCH-Head, \SEARCH-Section, \SEARCH-Link}

\begin{enumerate}
\item \SEARCH-Head falscher Attributname (topics statt topic)
\item \SEARCH-Section falscher Attributname (provide statt provider)
\item \SEARCH-Link falscher Attributname (license statt licence)
\item \SEARCH-Head Tag-Klammer vorne fehlt
\item \SEARCH-Section Tag-Klammer vorne fehlt
\item \SEARCH-Section Ende  Tag-Klammer hinten fehlt
\item \SEARCH-Section nicht geschlossen
\item \SEARCH-Link Beginn Tag-Klammer vorne fehlt
\item \SEARCH-Link Ende  Tag-Klammer hinten fehlt
\item \SEARCH-Link nicht geschlossen
\end{enumerate}

\section{Auswertung der Testfälle}

Sobald nur ein \SEARCH-Head vorhanden ist, wird dieser nicht ausgewertet und es werden keine \SEARCH-Lnks angezeigt.

Sobald nur eine oder mehrere \SEARCH-Sections vorhanden sind, werden diese nicht ausgewertet und es werden keine \SEARCH-Links angezeigt.

Sobald \SEARCH-Links vorhanden sind, werden diese entsprechend ausgewertet und sie werden korrekt in der TableView der\SEARCH-Tags angezeigt.

Es treten Fehler auf, sobald im \SEARCH-Link die Anführungszeichen fehlen, denn es werden trotzdem entsprechende \SEARCH-Links gefunden und fehlerhaft angezeigt.

Sobald \SEARCH-Tag-Klammern fehlen, können diese nicht ausgewertet werden und werden nicht angezeigt.