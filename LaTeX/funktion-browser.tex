\chapter{EEXCESS-Browser}

\section{Funktionalit�ten}

\begin{itemize}
	\item WebView
	\item Back/Forward Button
	\item Lesezeichen
	\item Lesezeichen hinzuf�gen
	\item Home-Button
	\item AddressBar
	\item Reload
	\item Men�
\end{itemize}

Mit Hilfe der WebView ist es m�glich, die gew�nschte Webseite
anzuzeigen. Befindet man sich auf einer Webseite mit SECH--Tags, so
erh�lt man zu diesen zus�tzliche Informationen.

Durch den Back beziehungsweise Forward--Button springt man eine Seite
zur�ck beziehungsweise vor.

In der AddressBar gibt man die URL der gew�nschten Webseite an. Nach
einem Klick in die WebView wird die URL gepr�ft und die Seite geladen.

Bei Eingabe eines Suchbegriffs zum Beispiel Bamberg wird man auf
\url{www.google.de} weitergeleitet.

Nach einem Klick auf das Lesezeichensymbol werden die Lesezeichen in einer
Tabelle angezeigt und k�nnen direkt dar�ber geladen werden.

�ber Lesezeichen hinzuf�gen kann man der aktuellen Website einen Namen geben
und sie unter diesem speichern.

Durch Klick auf den Home-Button wird man auf die aktuell festgelegte
Startseite weitergeleitet.

Durch den Reload-Button ist es m�glich, die aktuelle Webseite neu zu laden.