\chapter{\SECH-Browser}

\section{Funktionalitäten}

\begin{itemize}
	\item WebView
	\item Back/Forward Button
	\item Lesezeichen
	\item Lesezeichen hinzufügen
	\item Home-Button
	\item AddressBar
	\item Reload
	\item Menü
\end{itemize}

Mit Hilfe der WebView ist es möglich, die gewünschte Webseite
anzuzeigen. Befindet man sich auf einer Webseite mit \SEACH--Tags, so
erhält man zu diesen zusätzliche Informationen.

Durch den Back- beziehungsweise Forward--Button springt man eine Seite
zurück beziehungsweise vor.

In der AddressBar gibt man die URL der gewünschten Webseite an. Nach
einem Klick in die WebView wird die URL geprüft und die Seite geladen.

Bei Eingabe eines Suchbegriffs zum Beispiel Bamberg wird man auf
\url{www.google.de} weitergeleitet.

Nach einem Klick auf das Lesezeichensymbol werden die Lesezeichen in einer
Tabelle angezeigt und können direkt darüber geladen werden.

Über Lesezeichen hinzufügen kann man der aktuellen Website einen Namen geben
und sie unter diesem speichern.

Durch Klick auf den Home-Button wird man auf die aktuell festgelegte
Startseite weitergeleitet.

Durch den Reload-Button ist es möglich, die aktuelle Webseite neu zu laden.