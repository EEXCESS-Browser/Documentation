
%opening
\title{PopOver Funktion für die Search Tags}
\author{Andreas Netsch, Philipp Winterholler}

\section{PopOver Funktion}

Damit sich der Nutzer gefundene \SEARCH-Tags direkt in dem Browserfenster anzeigen lassen kann, wurde die PopOver Funktion eingeführt. Diese stellt den angeklicketen \SEARCH-Tag in einem neuen Fester auf dem Screen dar. Dabei ist es egal ob der Benutzer den \SEARCH-Tag auf der Website oder in der TableView anklickt.
Die folgende Funktion 
\begin{lstlisting}
    override func prepareForSegue(segue: UIStoryboardSegue, 
    sender: AnyObject?){
        ...
    }
\end{lstlisting}

erzeug den PopOverView und liefert den angeklicketen \SEARCH-Tag mit. Dieser wird dann als Überschrift wie in Abbildung 1, dargestellt. 
\begin{figure}[h]
    \centering
    \includegraphics[height=0.4\textheight]{PopOver}
    \caption{Liste für angeklickten \SEARCH-Tag}
    \label{fig:angeklickt}
\end{figure}
In dem PopOverView wird oben links der Titel des angeklickten \SEARCH-Tags dargestellt. Daneben wird ein Button dargestellt, der eine weitere TableView enthält. Rechts oben wird noch das Logo des aktuellen Browsers angezeigt. Durch einen Klick auf den Button \glqq Alle Suchergebnisse\grqq  wird eine neue TableView geöffnet, die weitere Suchergebnisse zu dem aktuellen Suchbegriff enthält (vgl. Abb. 16.8). Zum Öffnen der Tabelle wurde folgende Funktion auf den Button gelegt
\begin{lstlisting}
@IBAction func openTable(sender: AnyObject) {
        if let popover = popoverContent.popoverPresentationController {
            let viewForSource = sender as! UIView
            popover.sourceView = viewForSource
            popover.sourceRect = viewForSource.bounds
            popoverContent.preferredContentSize = CGSizeMake(400,500)
        }        
        self.presentViewController(popoverContent, animated: true,
         completion: nil)
    }
\end{lstlisting}
\begin{figure}[h]
    \centering
    \includegraphics[height=0.4\textheight]{Table_in_PopOver}
    \caption{Liste aller Suchergebnisse}
    \label{fig:Alle Suchergebnisse}
\end{figure}
Durch diese Funktion, hat der Benutzer die Möglichkeit sich unterschiedliche Ergebnisse zu einem \SEARCH-Tag innerhalb eines Fensters anzeigen zu lassen. Durch einen Tap auf ein anderes Ergebnis in der Tabelle, läd der darunter liegende Webview neu. Um das Popover Fenster wieder zu schließen, muss der Benutzer nur innerhalb des Views tippen. Dadurch wird wieder der normale WebView angezeigt.
