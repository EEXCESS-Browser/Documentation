\documentclass[a4paper,12pt]{article}
\usepackage[utf8]{inputenc}
\usepackage[german, ngerman]{babel}
\usepackage{graphicx}
\usepackage[figurename=Bild]{caption}
\usepackage{inputenc}
\usepackage{geometry}


%opening
\title{TeamUI_Aufgabenverteilung}
\author{Brian Mairhörmann}

\begin{document}

\section{Team:UI}

Teamleiter: Brian Mairhörmann\\
Teammitglieder: Andreas Netsch, Andreas Ziemer, Patrick Büttner, Philipp Winterholler\\
Das Team:UI ist für die benutzergerechte Bedienung und Darstellung der Search-Tag zuständig.
Das Team analysiert die Websites auf Search-Tag, bringt diese auf Programmebene und übergibt sie
zur Abfrage an das Team:Content. Anschließend werden die Antworten des Antworten des EEXCESS-Servers
auf der Benutzeroberfläche dargestellt.\\
Es folgt eine Aufzählung der Aufgaben, welche im Team:UI erledigt wurden, und deren bearbeitende Person/en:\\
\begin{itemize}
	\item System-/Softwarearchitektur - (Mairhörmann, Recum)
	\item Entwicklung der Search-(Sech-)Tags - (Mairhörmann, Recum)
	\item Vorbereitung des Layouts - (Mairhörmann, Recum)
	\item Implementierung der Adressleiste - (Ziemer)
	\item Implementierung des Einstellungsmenüs - (Ziemer)
	\item Implementierung der Lesezeichenfunktion - (Büttner)
	\item Entwicklung des Layouts des Browser - (Erol, Netsch, Winterholler)
	\item Einrichten der spezifischen(zum testen der Regex und Websiteanalyse) und allgemeinen(zum entwickeln der Software) Testwebsites - (Ziemer)
	\item Erstellung von spezifischen Testwebsites - (Büttner)
	\item RegEx für Websiteanalyse - (Mairhörmann)
	\item Auslesen der SearchTags aus HTML-Code - (Mairhörmann)
	\item Implementierung der TableView zur Anzeige von SearchTags und PopOverFunktion - (Netsch, Winterholler)
	\item Umstellung des Browsers auf UIWebKit - (Ziemer)
	\item Animation der TableView - (Erol)
	\item Constraints setzten - (Erol, Winterholler)
	\item Anzeige der Anzahl der Sechtags - (Winterholler)
	\item Visualisierung der SearchTags im Browser - (Mairhörmann, Netsch)
\end{itemize}




\end{document}
