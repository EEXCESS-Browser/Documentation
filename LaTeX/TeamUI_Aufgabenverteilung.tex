
\section{Team:UI}

Teamleiter: Brian Mairh�rmann\\
Teammitglieder: Andreas Netsch, Andreas Ziemer, Patrick B�ttner, Philipp Winterholler\\
Das Team:UI ist f�r die benutzergerechte Bedienung und Darstellung der Search-Tag zust�ndig.
Das Team analysiert die Websites auf Search-Tag, bringt diese auf Programmebene und �bergibt sie
zur Abfrage an das Team:Content. Anschlie�end werden die Antworten des Antworten des EEXCESS-Servers
auf der Benutzeroberfl�che dargestellt.\\
Es folgt eine Aufz�hlung der Aufgaben, welche im Team:UI erledigt wurden, und deren bearbeitende Person/en:\\
\begin{itemize}
	\item System-/Softwarearchitektur - (Mairh�rmann, Recum)
	\item Entwicklung der Search-(Sech-)Tags - (Mairh�rmann, Recum)
	\item Vorbereitung des Layouts - (Mairh�rmann, Recum)
	\item Implementierung der Adressleiste - (Ziemer)
	\item Implementierung des Einstellungsmen�s - (Ziemer)
	\item Implementierung der Lesezeichenfunktion - (B�ttner)
	\item Entwicklung des Layouts des Browsers - (Erol, Netsch, Winterholler)
	\item Einrichten der spezifischen(zum Testen der Regex und Websiteanalyse) und allgemeinen(zum entwickeln der Software) Testwebsites - (Ziemer)
	\item Erstellung von spezifischen Testwebsites - (B�ttner)
	\item RegEx f�r Websiteanalyse - (Mairh�mann)
	\item Auslesen der SearchTags aus HTML-Code - (Mairh�rmann)
	\item Implementierung der TableView zur Anzeige von SearchTags und PopOverFunktion - (Netsch, Winterholler)
	\item Umstellung des Browsers auf WKWebKit - (Netsch, Ziemer)
	\item Animation der TableView - (Erol, Winterholler)
	\item Constraints setzten - (Erol, Winterholler)
	\item Anzeige der Anzahl der Sechtags - (Winterholler)
	\item Visualisierung der SearchTags im Browser - (Mairh�rmann, Netsch)
\end{itemize}
