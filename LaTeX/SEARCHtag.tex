\author{Gottfried von Recum}
\chapter{\SEARCH-Tag}

\section{Allgemeines}
Ein \SEARCH-Tag gibt dem Ersteller einer HTML5 Seite die Möglichkeit seinen Content dynamisch anzureichern. Dazu kann er einige Attribute definieren.
Ein \SEARCH-Tag besteht aus bis zu drei Teilen, die an verschiedenen Stellen in den HTML5 Quellcode der Seite eingefügt werden können.
Die generelle Form ist
\begin{lstlisting}
	<search attribute="">...</search>
\end{lstlisting}

Die drei Teile sind angelehnt an HTML5:
\begin{lstlisting}
	<search-head>
	<search-section>
	<search-link>
\end{lstlisting}

Jeder Teil eines \SEARCH-Tags hat stets das Attribut \Verb|topic="text"| wobei \Verb|text| ein frei wählbarer Begriff ist, der zur Contentgenerierung verwendet werden soll.

Die vier optionalen Filter Attribute \Verb|type=""|, \Verb|mediaType=""|, \Verb|provider=""| und \Verb|licence=""| dienen der Optimierung der Suchanfrage und anschließenden Antwortauswahl.

\section{\SEARCH-Head}
[optional]
Der \Verb|<search-head>| Teil wird \textbf{optional} in den Header der Seite eingefügt um allgemeine Attribute für die Seite als Ganzes definieren zu können.

\begin{lstlisting}
<head>
    <search-head topic="text">
</head>
\end{lstlisting}

oder

\begin{lstlisting}
<head>
    <search-head topic="Programmierer" type="misc" mediaType="unknown"
    provider="ZBW" licence="unknown">
</head>
\end{lstlisting}

\section{\SEARCH-Link}
[obligtorisch]
Der \textbf{obligatorische} \Verb|<search-link>| Teil eines \SEARCH-Tags umschließt einen Bereich, der mit Inhalt angereichert werden soll. Dieser kann hervorgehoben dargestellt werden.

\begin{lstlisting}
<body>
    <search-link topic="text">...</search-link>
</body>
\end{lstlisting}

oder

\begin{lstlisting}
<body>
    <search-link topic="Ada" type="person" mediaType="image"
    provider="ZBW" licence="unknown">Ada Lovelace</search-link>
</body>
\end{lstlisting}

\section{\SEARCH-Section}
[optional]
Der \Verb|<search-section>| Teil wir \textbf{optional} im Body des HTML5 Quellcodes platziert und umschließt einen oder mehrere \Verb|<search-link>| Teile. Damit kann dem \SEARCH-Link Teil ein erweiterter Themenbezug beiseite gestellt werden.

\begin{lstlisting}
<body>
    <search-section topic="text">
    ...
    </search-section>
</body>
\end{lstlisting}

oder

\begin{lstlisting}
<html>
<head>
    <search-meta topic="Programmierer" type="misc" mediaType="unknown"
    provider="ZBW" licence="unknown">
</head>
<body>
    ...
    <search-section topic="Grossbritannien" type="location"
    mediaType="unknown" provider="ZBW" licence="unknown">
    ...
    <search-link topic="Ada" type="person" mediaType="image"
    provider="ZBW" licence="unknown">Ada Lovelace</search-link>
    ...
    </search-section>
</body>
</html>
\end{lstlisting}

\section{\SEARCH-Tag}
Der \SEARCH-Tag schließlich besteht aus allen drei Teilen (soweit vorhanden), dabei wird der \SEARCH-Link Teil am stärksten gewichtet.

Dabei akkumulieren die \Verb|topics| und \Verb|types|.
Die \Verb|mediaTypes|, \Verb|provider| und \Verb|licences| hingegen ersetzen sich mit zunehmender Priorität in Richtung \SEARCH-Link.

\section{Klassen Implementierung}
Es entsteht ein Objekt der Klasse \Verb|search|:

\begin{lstlisting}
class Search{
    var id = String()
    var response = String() //Tbd
    var detail = String() //Tbd
    var tags = [String : Tag]()
    // String is id (link, section, head) and Tag is Tag-Object
    var filters = Filter()
}

class Tag {
    var topic = String()
    var type = String()
    var isMainTopic = false
}

class Filter{
    var mediaType = String()
    var provider = String()
    var licence = String()
}
\end{lstlisting}