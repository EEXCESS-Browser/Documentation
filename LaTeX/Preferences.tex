
\chapter{Einstellungsmöglichkeiten}

Der Nutzer hat die Möglichkeit bestimmte Einstellungen vorzunehmen, welche die Ergebnisse der Suchanfragen beeinflussen.   

Folgende Einstellungen können getätigt werden:

\begin{itemize}
	\item Geschlecht des Nutzers (männlich, weiblich)
	\item vom Nutzer bevorzugte Sprache (deutsch, englisch)
	\item Stadt
	\item Land
	\item Geburtsdatum des Nutzers
\end{itemize}


Nachdem der Nutzer alle Einstellungen getätigt hat, werden die entsprechenden Attribute in
ein Dictionary der Form [String:String] hinzugefügt und die Methode ''createJSONForRequest
(keyWordsWithKeys:[String:AnyObject],
detail:Bool, pref:[String:String]) -\textgreater JSONObject?'' der Klasse ''MainController'' wird aufgerufen. Diese Methode berechnet zu erst mit Hilfe der Methode ''calculateAgeRange(ageInYears:Int) -\textgreater Int'' die sog. ''Age-Range''. Die ''Age-Range'' ist eine als String gespeicherte Zahl und gibt an ob es sich beim Nutzer um ein Kind, jungen Erwachsenen oder Erwachsenen handelt. Welche ''Age-Range'' jedem Alter zugeordnet ist, kann folgender Tabelle entnommen werden:

\begin{tabular}{c|c|c}
Alter & Bezeichnung & Age-Range \\
\hline
0-17 & Kind & 0 \\
18-25 & junger Erwachsener & 1 \\
ab 26 & Erwachsener & 2
\end{tabular}

Danach wird das entsprechende JSON-Objekt mit den dazugehörigen Attributen erzeugt.

